\documentclass{article}
\usepackage[utf8]{inputenc}
\usepackage[spanish]{babel}
\usepackage{listings}
\usepackage{graphicx}
\graphicspath{ {images/} }
\usepackage{cite}

\begin{document}

\begin{titlepage}
    \begin{center}
        \vspace*{1cm}
            
        \Huge
        \textbf{Ideación juego}
            
        \vspace{0.5cm}
        \LARGE
        Proyecto final
            
        \vspace{1.5cm}
            
        \textbf{Diego Alejandro Osorio Jimenez }
            
        \vfill
            
        \vspace{0.8cm}
            
        \Large
        Despartamento de Ingeniería Electrónica y Telecomunicaciones\\
        Universidad de Antioquia\\
        Medellín\\
        Marzo de 2021
            
    \end{center}
\end{titlepage}

\tableofcontents
\newpage
\section{Introducción}\label{intro}
Se plasmarán las primeras ideas relacionadas con el juego a desarrollar, pensado como actividad evaluativa del conocimiento adquirido en el curso de informática 2, es importante entender que aún estamos empezando con conceptos de programación lo cual podría influir en las altas o bajas expectativas que se puedan tener del juego, ya que no tenemos las herramientas necesarias aún para definir hasta qué punto podemos trabajar en este. 

\newpage
\section{Concepto del juego } \label{contenido}
Lo primero que se me vino a la cabeza es; "¿Qué tipo de juego quiero desarrollar?", "¿Cuál será su mayor característica?". Empecé a investigar un poco y concluí que quería un juego de rol por sus siglas en ingles RPG (role playing game) donde su valor agregado fuera una buena trama, que pudiera cimentar sus bases en una historia, que, aunque corta; por la misma construcción del juego, pueda llamar la atención de quien decida jugarlo y lo mantenga entretenido hasta el final de este mismo. Así; puedo combinar dos gustos, la programación y la escritura.
Ahora como segundo paso me queda empezar a idearme la construcción del guion, ya que pienso que este es el que va a dar las bases para definir los objetos, escenarios y demás atributos del juego, para que así; este pueda tener una coherencia entre historia y parte visual. Ya la parte del código se ira haciendo más clara a medida que avancemos en el curso y por cuenta propia pueda adquirir mayor conocimiento de las herramientas que pueda utilizar en este lenguaje de programación y específicamente el entorno de desarrollo QT.
Inicialmente tengo la idea de trabajar solo, ya que enfrentarme a los retos que el desarrollo de esta actividad pueda traer, puede formar un buen carácter y aumentar mis habilidades de programación de manera satisfactoria, pero no le cierro la posibilidad de también fortalecer el trabajo en equipo y departir conocimientos con algún compañero que tenga las mismas ganas de aprender de esta actividad y lineamientos parecidos con el tipo de juego y objetivo de este. 








\end{document}
